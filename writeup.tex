\documentclass{article}
%if i need images \usepackage{graphicx}

\usepackage{listings}
\usepackage{color}

\definecolor{dkgreen}{rgb}{0,0.6,0}
\definecolor{gray}{rgb}{0.5,0.5,0.5}
\definecolor{mauve}{rgb}{0.58,0,0.82}

\lstset{frame=tb,
  language=Python,
  aboveskip=3mm,
  belowskip=3mm,
  showstringspaces=false,
  columns=flexible,
  basicstyle={\small\ttfamily},
  numbers=none,
  %numberstyle=\tiny\color{gray},
  %keywordstyle=\color{blue},
  %commentstyle=\color{dkgreen},
  %stringstyle=\color{mauve},
  breaklines=true,
  breakatwhitespace=true,
  tabsize=3
}


\begin{document}

\title{Grammar Analyzer Writeup - CS311 - Leblanc}
\author{Ben Reichert}

\maketitle

%\section{Introduction}

%Your writeup should walk through each of the above steps and explain why you took the approach
%you did. I would also like answers to the following questions.
%1. Why were those particular restrictions placed on the grammars?
%2. If those restrictions weren’t there, how would your program need to change?
%3. What would happen if the grammar were ambigious?
%4. What other approaches might be used to tell if a string can be generated by a given grammar?

%if you give the application a variable in the input string it infinite loops because it consistenly pushes and pops from teh stack indefinitely with teh variable provided, so dont do that:w

%\begin{lstlisting}
%\end{lstlisting}

\end{document}
